\chapter{Advanced programming in Lynx}
\label{chap:advancedprogramming}

In this chapter we describe how to design advanced functionalities in Lynx.

\section{Performance measures}

\subsection{Understanding containers}

Before implementing a performance measure, you need to understand the structure of the \verb|ValueContainer| object. A \verb|ValueContainer| is an object used to store several elements of a particular type (e.g. numbers or percentages), and to return an ``average'' value on request. Every performance measure should derive from the abstract class \verb|PerformanceMeasure|, and from a suitable \verb|ValueContainer| object. For example, the mean-squared error derives from \verb|NumericalContainer|; the misclassification error derives from \verb|PercentageContainer|; the ROC curve derives from \verb|XYPlotContainer|.

\subsection{Implementing performance measures}

The main method of a performance measure is the \verb|compute| method:

\begin{lstlisting}
perf = compute(obj, true_labels, predictions, scores);
\end{lstlisting}

\noindent Here, \verb|true_labels| are the targets values, \verb|predictions| are the predictions of a model, and \verb|scores| are the confidence scores. \verb|perf| is the output value of the performance measure, which should be compatible with the chosen \verb|ValueContainer| object. For example, a class deriving from \verb|NumericalContainer| should output numerical values; a class deriving from \verb|XYPlotContainer| should output \verb|XYPlot| objects; and so on. We also need to implement two additional methods:

\begin{itemize}
\item \verb|isCompatible(t)| checks that the performance measure is compatible with the task \verb|t|.
\item \verb|isBetterThan(obj, p))| checks if the current performance measure is better than performance measure \verb|p|.
\end{itemize}

Moreover, performance measures that are not comparable should set the internal property \verb|isComparable| to false in the constructor.

\section{Partition strategies}

Partitioning strategies must be extend the \verb|PartitionStrategy| class, and they can be placed inside the “\textit{functionalities/PartitionStrategies}” folder. The constructor of a partition strategy can have any number of parameters. The strategy must implement four methods:

\begin{itemize}

\item \verb|obj = partition(obj,Y)| is used to request a partition of the vector \verb|Y|. Each partition can define any number of splits of \verb|Y|, whose number must be stored in the property \verb|num_folds|.
\item \verb|ind = getTrainingIndexes(obj)| and \verb|ind = getTestIndexes(obj)| return the training and testing indexes of the current fold. The current fold can be retrieved from the property \verb| current_fold|. Indexes are $N \times 1$ vectors of logical elements, where $N$ is the dimension of the $Y$ vector of the previous method.
\item \verb|s = getFoldInformation(obj)| returns a string with information on
the current fold. This is used for printing information on the console during the simulation.
\end{itemize}

\section{Statistical tests}

A statistical test must extend the class \verb|StatisticalTest|. It requires the implementation of three static methods:

\begin{itemize}
\item \verb|[b, res] = check_compatibility(algorithms, datasets)| is used
to check the compatibility of the procedure with the chosen datasets
and algorithms. \verb|b| is a boolean indicating whether there is or not
compatibility. If \verb|b| is false, \verb|res| is a string describing the source of error.
\item \verb|perform_test(datasets_names, algorithms_names, errors)| performs
the test and prints the information on screen. The first two arguments are cell arrays containing the names of the datasets and of the algorithms, while the third argument is a $A \times D$ matrix of errors, where $A$ is the number of algorithms and $D$ is the number of datasets. The $ij$-th element of errors is the averaged error of the $i$-th algorithm on the $j$-th dataset.
\item \verb|s = getDescription| returns a string describing the statistical test.
\end{itemize}

\section{Tasks and factories}

Any task in Lynx derives from \verb|BasicTask|. Loading a dataset in Lynx proceeds as follows:

\begin{itemize}
\item Every task has an associated set of folders that contains datasets of the corresponding type. Starting from the filename, the dataset is searched from all the basic tasks, and associated to the first task containing a .mat file with the corresponding name.
\item The .mat file is checked for consistency using the \verb|checkForConsistency| method of the corresponding class.
\item One or more datasets are created using a \verb|DatasetFactory| object.
\end{itemize}

\noindent For example, \verb|RegressionTask| defined a regression task, which is loaded using the \verb|DummyFactory| object. The \verb|PredictionTask|, instead, is loaded using the \verb|EmbedTimeseriesFactory|, that transform it into a regression task by embedding the timeseries. You can have more than one factory associated to a particular class, and you can change them in the configuration file using the \verb|set_factory| method.

In the constructor of a task, you must define three properties:

\begin{itemize}
\item \verb|folders| is a cell array of folders containing datasets of this type.
\item \verb|performance_measure| is the default performance measure for the task (empty if not defined).
\item \verb|dataset_factory| is the default \verb|DatasetFactory| object associated to the task.
\end{itemize}

\noindent Together with the \verb|checkForConsistency| method, you must define two additional methods:

\begin{itemize}
\item \verb|s = getDescription(obj)| returns a string describing the task.
\item \verb|id = getTaskId(obj)| returns the id of the task, which should be contained in the \verb|Tasks| enumeration.
\end{itemize}

\noindent Implementing a \verb|DatasetFactory| object, instead, consists in a single method:

\begin{lstlisting}
datasets = create(obj, task, dID, data_name, o);
\end{lstlisting}

\noindent where \verb|task| is the task id, \verb|dID| and \verb|data_name| are the id and name given to the dataset by the user, and \verb|o| is a structure array with the content of the .mat file. \verb|datasets| is a cell array of \verb|Dataset| objects.

\section{Additional features}

Additional features can derive from the abstract \verb|AdditionalFeature| class. They can implement one or more of its methods, that act at predefined moments during the simulation. For additional information, see the help of \verb|AdditionalFeature|.